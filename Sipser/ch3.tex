\chapter{Context Free Languages}

\section{Context-Free Grammers}

Example of a context-free grammer we will call $G_1$:

\begin{align*}
    A &\rightarrow 0A1 \\
    A &\rightarrow B \\  
    B &\rightarrow \# \\
\end{align*}    

Each grammer consists of a collection of \textbf{substitution rules}, also
called \textbf{production}. Each rule appears as a line in the grammer, 
comprising a symbol and a string separeated by an arrow. The symbol is called
a \textbf{variables}. The string consists of variables and other symbols called
the \textbf{terminals}. Often the variable symbols are captial letters. terminals
are similar to alphabet and are often represented by lowercase letters, numbers,
or special symbols.

You use a grammer to describe a language by generating each string of that
language as follows:

\begin{enumerate}
    \item Write down the start variable. Usually the left-hand side top rule.
    \item Find a variable written down and a rule starting witht hat variable.
        Replace the written down variable with the right-hand side of that rule.
    \item Repeat step 2 until no varialbes remain.
\end{enumerate}

For example, 

$$
A \rightarrow 0A1 \rightarrow 00A11 \rightarrow 000B111 \rightarrow 000\#111
$$

All strings generated in this way constitute the \textbf{language of the grammer}.

\subsection{Formal Defination Of A Context-Free Grammer} 

\begin{defination}
    A \textbf{context-free grammer} is a 4-tuple $(V, \sum, R, S)$, where
    \begin{enumerate}
        \item $V$ is a finite set called \textbf{variables}.
        \item $\sum$ is a finite set, disjoint from $V$, called the \textbf{terminals},
        \item $R$ is a finite set of \textbf{rules}, with each rule beign
            a variable and a string of variables and terminals, and
        \item $S \in V$ is the start variables
    \end{enumerate}
\end{defination}

If $u, v$ and $w$ are string of variables and terminals, and $A \rightarrow w$
is a rule of the grammer, we say that $uAv$ \textbf{yields} $uwv$, written
$uAv \implies uwv$. Say that $u$ \textbf{derives} $v$m written
$u \derives$, if $u = v$ or if a sequence $u_1, u_2, \dots, u_k$ exists for
$k \ge 0$ and 

$$
u \implies u_1 \implies u_2 \implies \dots \implies u_k \implies v
$$

The \textbf{language of the grammer} is $\{w \in \sum^{*} \mid S \derives w\}$

\subsection{Ambiguity}

If a grammer generates the same string in several different ways, we say that
the string is derived \textit{ambiguously} in that grammer. If a grammer 
generates some string ambigously, we say that the grammer is \textit{ambiguous}.

When we say that a grammer generates a string ambiguously, we mean that
the string has two different parse trees, not two different derivations. 
Two derivations may differ merely in the order in which the replace variables
and not the overall strucutre. to concentrate on structure, we define a type of
derivation that replaces variables in a fixed order. A derivation of a string
$w$ in a grammer $G$ is a \textbf{leftmost derivation} if at every step the
leftmost remaining variable is the one replaced.

\begin{defination}
    A string $w$ is derived \textbf{ambiguously} in a context-free grammer $G$
    if it has two or more different leftmost derivations. Grammer $G$ is 
    \textbf{ambiguous} if it generates some string ambiguously.
\end{defination}

Some context-free languages can be generated only by ambiguous grammers. Such
languages are \textbf{inherently ambiguous}.

\subsection{Chomsky Normal Form} 

\begin{defination}
    A context-free grammer is in \textbf{Chomsky normal form} if every rule
    is of the form:

    \begin{align*}
        A &\rightarrow BC \\
        A &\rightarrow a
    \end{align*}

    where $a$ is any terminal and $A, B, and C$ are any variables. except that
    $B$ and $C$ may not be the start variable. In addition, we permit the rule
    $S \rightarrow \epsilon$, where $S$ is the start variable.
\end{defination}
