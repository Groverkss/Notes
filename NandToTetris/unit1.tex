\chapter{Boolean Functions and Gate Logic}

\section{Boolean Logic}

\subsection{Boolean Functions Synthesis}

Given a Truth Table, how do we construct a boolean function for it?\\

Lets take an example, 

\begin{table}[h!]
    \begin{center} 
        \caption{Truth Table for a Boolean Function}
        \label{tab:example1}
        \begin{tabular}{l|l|l|l}
            $x$ & $y$ & $z$ & $f$\\
            \hline
            0 & 0 & 0 & 0\\
            0 & 0 & 1 & 1\\
            0 & 1 & 0 & 0\\
            0 & 1 & 1 & 0\\
            1 & 0 & 0 & 0\\
            1 & 0 & 1 & 0\\
            1 & 1 & 0 & 1\\
            1 & 1 & 1 & 0\\
        \end{tabular}
    \end{center}
\end{table}

This table can be represented by taking OR of the "true"" statements. 
We can represent the "true" statements by a boolean expression.
For example, $x = 0, y = 0, z = 1, f = 1$ can be represented by
$\neg x \land \neg y \land z$. This expression is only true when
$x = 0, y = 0, z = 1$ and false on all other cases. So, we can
represent every boolean function by AND of such terms.\\

The truth table \ref{tab:example1} can be represented by the expression:

$$
(\neg x \land \neg y \land z) \lor (x \land y \land \neg z)
$$

But what is the minimum size expression we can build from a truth
table? This is a NP-Complete problem and cannot be solved in
polynomial time if $P \neq NP$.\\

\subsection{Why NAND?}
We can represent every boolean expression only using NAND. This
can be trivially proved.

\section{Gate Logic}

A technique for implementing Boolean functions using logic gates.\\

Logic Gates:
\begin{enumerate}
    \item Elementary (Nand, And, Or, Not)
    \item Composite (Mux, Adder)
\end{enumerate}

\section{Hardware Description Language}

Here is a possible implementation of a XOR gate in HDL.

\begin{lstlisting}
// Xor gate: out = (a And Not(b)) Or (Not(a) And b)

CHIP Xor {
    IN a, b;
    OUT out;

    PARTS:
    Not (in=a, out=nota);
    Not (int=b, out=notb);
    And (a=a, b=notb, out=aAndNotb);
    And (a=nota, b=b, out=notaAndb);
    Or (a=aAndNotb, b=notaAndb, out=out);
}
\end{lstlisting}

\subsection{Some comments on HDL}

\begin{itemize}
    \item HDL is a functional / declarative language
    \item The order of HDL statements is insignificant
    \item Before using a chip part, you must know its interface.
        For example: 
        \begin{lstlisting}
        Not(in= ,out= ), And(a=, b= ,out= )
        \end{lstlisting}
    \item Connections like
        \begin{lstlisting}
        partName(a=a,...) partName(..., out=out)
        \end{lstlisting}
        are common
\end{itemize}

Common HDLs:

\begin{itemize}
    \item VHDL
    \item Verilog
    \item Many more HDLs
\end{itemize}

Our HDL:

\begin{itemize}
    \item Similar in spirit to other HDLs
    \item Minimal and simple
    \item Provides all you need for this course
\end{itemize}

\section{Multi-Bit Buses}

Sometimes we manipulate "together" an array of bits.
It is conceptually convenient to think about such a group of bits
as a single entity, sometimes termed "bus".
HDLs will usually provide some convenient notion for handling
these buses.

\subsection{Examples on buses in HDL}

Here is and example of how buses work in HDL:

\begin{lstlisting}
/*
* Adds three 16-bit values
*/

CHIP Add3Way16 {
    IN first[16], second[16], third[16];
    OUT out[16];

    PARTS:

    Add16(a=first, b=second, out=temp);
    Add16(a=temp, b=third, out=out);
}
\end{lstlisting}

\subsection{Sub-buses}

Buses can be composed from (and broken into) sub-buses:

\begin{lstlisting}
...
IN lsb[8], msb[8], ...
...
Add16(a[0..7]=lsb, a[8..15]=msb, b=..., out=...);
Add16(..., out[0..3]=t1m out[4..15]=t2);
\end{lstlisting}

Some syntactic choices of our HDL:

\begin{itemize}
    \item Overlaps of sub-buses are allowed on output buses of parts
    \item Width of internal pins is deduced automatically
    \item "false" and "true" may be used as buses of any width
\end{itemize}

\section{Prespectives}

In this chapter we built a basic chipset of 15 chips and we will
use these to build more advanced chips in later chapters.\\

We use NAND gates in industry because NAND gates are cheap to make. But since we are looking at lower levels and then creating abstractions, lets see how a NAND gate is made.

\begin{figure}[h!]
    \begin{center}
        \begin{circuitikz}
            \draw (0, -2)
            node[nmos] (nmosA) {}
            (nmosA.G) to[short, -o] ++(0, 0) node[left] {$A$} (0, -4)
            node[nmos] (nmosB) {}
            (nmosB.G) to[short, -o] ++(0, 0) node[left] {$B$}
            (nmosA.S) to (nmosB.D)
            (nmosB.S) node[ground] {}
            (nmosA.D) to (0, 0) node[vcc] {$V_{cc}$}
            to[short, *-o] ++(1, 0) node[right] {out};
        \end{circuitikz}
        \caption{A NMOS Nand Gate}
    \end{center}
\end{figure}

If A and B, both are ON, then the current goes from $V_{cc}$ to Ground, and thus $out = 0$ or it goes to out, implying $out = 1$.
