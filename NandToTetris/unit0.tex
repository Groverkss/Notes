\chapter{Introduction}

\section{The Big Picture}

In Part-1 we will build hardware of the computer. In Part-2 we will complete 
the picture and build the software heirarchy of the computer.

\subsection{The Road Ahead}
How do you actually print "Hello World"? Not writing code for it, but how does
it actually work? Why don't we have to worry about it?
We only care about "what" is to be done. \\

"How" $\leftarrow$ Implementation \quad
"What" $\leftarrow$ Abstraction \\

But who will worry about the "how"? Someone has to do it. A nice thing about
computers is once we have done the "how" we only need to
worry about the "what". \\

\subsection{Multiple Layers of Abstraction}
Once we have built the lower level, we dont need to
worry about it and can abstract it. \\

Every week, we will worry about a single level, take the lower level as given,
implement the higher level and test that it works. \\\\

By the end of the course, we will have built a complete functioning computer
and can run anything including games like Tetris.

\subsection{Two Parts}
\begin{enumerate}
    \item Part-I : Hardware
        \begin{enumerate}
            \item Start with \textit{Nand}
            \item Create the \textit{HACK} computer
        \end{enumerate}
    \item Part-II : Software
        \begin{enumerate}
            \item Start with the \textit{HACK} computer
            \item Create a fill sotware hierarchy that $\dots$
            \item $\dots$ runs applications like \textit{Tetris}
        \end{enumerate}
\end{enumerate}

\subsection{From Nand To Hack}

Nand
$\xrightarrow{\text{Combinational Logic}}$ Elementary Logic Gates
$\xrightarrow{\text{Comb. and Seq. Logic}}$ CPU, RAM, chipset
$\xrightarrow{\text{Digital Design}}$ Computer Architecture
$\xrightarrow{\text{Assembler}}$ Low Level Code

\subsection{How to build a chip}

We will use build our chip on a hardware simulator. We will do
this in a HDL (Harware Discription Language).
