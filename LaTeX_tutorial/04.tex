\documentclass{article}

% Using math equation in latex

\usepackage{amsmath}

\begin{document}
The formula $f(x) = x^2$ is an example.

% You cannot type more than 1 equation in equation environment

% Using equation*
\begin{equation*}
    1 + 2 = 3
\end{equation*}

\begin{equation*}
    1 = 3 - 2
\end{equation*}

% You can type more than 1 equation in align mode. The * is to disable
% numbering for each equation

% Using align*
\begin{align*}
    1 + 2 &= 3\\ % \\ seperats equations by a newline
    1 &= 3 - 2   % Equations are aligned according to &=
\end{align*}

% Usage of amsmath
\begin{align*}
    f(x) &= x^2\\
    g(x) &= \frac{1}{x}\\
    F(x) &= \int^a_b \frac{1}{3}x^3\\
    G(x) &= \frac{1}{\sqrt{x}}\\
\end{align*}

% Matrix in display math mode
$$
\begin{pmatrix}
    0 & 1\\
    1 & 0
\end{pmatrix}
$$

% Matrix in inline math mode
This is a matrix $
\begin{pmatrix}
    0 & 1\\
    1 & 0
\end{pmatrix}
$ in inline math mode.

Putting non-scaling brackets $\rightarrow (\frac{1}{\sqrt{x}})$ vs
putting scaling brackets $\rightarrow \left(\frac{1}{\sqrt{x}}\right)$.
\end{document}
