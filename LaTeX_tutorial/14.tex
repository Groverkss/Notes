\documentclass{article}

\usepackage{tikz}
\usepackage{circuitikz}
\usetikzlibrary{positioning}

\begin{document}

\begin{figure}[h!]
    \begin{center}
        \begin{circuitikz}
            \draw (0, 0)
            to[V, v=$U_q$] (0, 2) % The voltage source
            to[short] (2, 2)
            to[R=$R_1$] (2, 0) % The resistor
            to[short] (0, 0);
        \end{circuitikz}
        \caption{Circuitikz bois}
    \end{center}
\end{figure}

\begin{figure}[h!]
    \begin{center}
        \begin{circuitikz}
            \draw (0, 0)
            to[V, v=$U_q$] (0, 2) % The voltage source
            to[short] (2, 2)
            to[R=$R_1$] (2, 0) % The resistor
            to[short] (0, 0);

            \draw(2, 2) 
            to[short] (4, 2)
            to[L=$L_1$] (4, 0)
            to[short] (2, 0);

            \draw(4, 2)
            to[short] (6, 2)
            to[C=$C_1$] (6, 0)
            to[short] (4, 0);
        \end{circuitikz}
        \caption{Bigger Circuitikz bois}
    \end{center}
\end{figure}

\begin{figure}[h!]
    \begin{center}
        \begin{circuitikz}
            \draw 
            (0, 2) node[and port] (myand) {}
            (2, 1) node[or port] (myor) {}
            (myand.in 1) node[anchor=east] {A}
            (myand.in 2) node[anchor=east] (bnode) {B}
            (myand.out) -| (myor.in 1)
            node[below=of bnode] (cnode) {C}
            (cnode) -| (myor.in 2);
        \end{circuitikz}
        \caption{Logic Gates}
    \end{center}
\end{figure}

\end{document}
