\chapter{Lecture 1}

\section{Course Details}

Textbook: Elmasri\\

Contents of Course:
\begin{enumerate}
\item Introduction/Database Concepts
\item Data Modeling
\item Relational Model
\item Normalization
\item Algebra/Calculas/SQL
\end{enumerate}

\subsection{Project}

Database Design Project.\\

-- Four phases - Rotation
-- group of 3 members

1 -- Requirements Analusis
2 -- Conceptual Design
3 -- Logical Design 
4 -- Applications \& SQL

\subsection{Motivation for the course}

\begin{itemize}
\item As a CS person, you are going to work with data and databases.
\item Second Largest Software Sales is Database Systems
Want a job? Do this course
\item A lot more was said but i wasnt paying attention :( 
    Overall a useful course.
\end{itemize}

\section{What are Database Systems?}

Any piece of information that can be captured is data. \\

There is birth of data and rarely a death of data.
For example, you send a ping to another computer and ask "Are you alive?", the time and message is recorded. The time and message is data. The time is the birth of data.\\

Data $\leftarrow$ Factual (undoubted) information that can be
recorded and have implicit meaning. \\

A database is a collection of related data. \\

For example, what courses students of a batch are taking.
This is related data. A database can be formed from this data and
probably already exists in IIIT database.\\

\subsection{What is a Database?}

A database has the following implicit properties:

\begin{itemize}
\item A database represents some aspect of the real world. (Universe of Discourse)
\item A database is a logically coherent (associated, related) collection of data with some inherent meaning.
\item A database is designed, build, and populated with data for a specific purpose.
\item It has an intended group of users and some preconceived (already thought of) applications in which these users are interested.
\end{itemize}

\subsection{Database System}
A database system (DBMS) is a collection of programs that enables
users to create and maintain a database.

\begin{itemize}
\item Defining Databases -- involves specifying the data types,
    structures, and contraints for the data to be stored in the
    database.
\item Constructing Databases -- sotring the data itself (populating)
    on some storage medium that is controlled by the DBMS.
\item Manipulating Databases -- quering the database to retrieve specific data, updating the databases to reflect changes to mini-world.
\end{itemize}

\subsubsection{Simplified Database System}

(Figure)

\subsection{Example of a Database}

Consider a part of a University environment\\

We need data about:
    STUDENTSs\\
    COURSEs\\
    SECTIONs (of COURSEs)\\
    (academic) DEPARTMENTs\\
    INSTRUCTORSs\\

The above data is related as follows:
    SECTIONs are of specific COURSEs\\
    STUDENTs take SECTIONs\\
    COURSEs have prequistic COURSEs\\
