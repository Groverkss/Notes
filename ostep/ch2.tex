\chapter{CPU Virtualisation}

\section{The Abstraction: The Process}

\textbf{Process:} A running program.\\

\textbf{Crux Of The Problem:} Although there are only a few physical CPUs
available, how can the OS provide the ullusion of nearly-endless supply of 
said CPUs?\\

The OS creates this illusion by \textbf{virtualizing} the CPU. By running
one process, then stopping it and running another, and so forth, the OS can
promote the illusion that many virtual CPUs exist when in fact there is only
one physical CPU (or a few). This technique is called \textbf{time-sharing}.\\

To do this, OS requires some low-level machinery and some
high-level intelligence. The low-level machinery is called \textbf{mechanisms}.
For example, stopping one program and starting another on a given CPU. On top
of these mechanisms, there is high-level intelligence call \textbf{policies}.
Policies are algorithms for making decisions within the OS.\\

Mechanisms: Provides answer to a \textit{how} question.

Policies: Provides answer to a \textit{which} question.\\

\section{Mechanism: Limited Direct Execution}

\textbf{The Crux: How to efficiently virtualize the CPU with control?}

The OS must virtualize the CPU in an efficient manner while retaining control
over the system. To do so, both hardware and operating-system support will be
required. The OS will often use a judicious bit of hardware support in order to
accomplish its work effectively.

\subsection{Basic Technique: Limited Direct Execution}

The \textit{direct execution} part of the idea: just run the program directly
on the CPU. Thus, when the OS wishes to start a program, it creates a process
entry for it in a process list, allocates some memory for it, loads the program
code into memory, locates its entery point (main()), jumps to it, and starts
running the user's code.\\

This approach gives rise to a few problems:

\begin{itemize}
    \item How to make sure that the program does not do anything that we dont
        want it to do while still maintaining efficency?
    \item How do we stop a process and switch to another process, i.e. how
        to implement \textbf{time sharing} we require to virtualize the CPU?
\end{itemize}

\subsubsection{Problem \#1: Restricted Operations}

What if the process wishes to perform some kind of restricted operation, such
as issuing an I/O request to a disk, or gaining access to more system resources
such as CPU or memory?\\

\textbf{The Curx: How to perform restricted operations}

A process must be able
to perform I/O and some other restricted operations, but without giving the
process complete control over the system.\\

Two differend modes:

\begin{itemize}
    \item \textbf{User Mode:} Code that runs in user mode is restricted in what
        it can do. For example, when running in user mode, a process can't
        issure I/O requests; doing so would result in the processor raising an
        exception.
    \item \textbf{Kernal Mode:} Mode the operating system (or kernal) runs in.
        In this mode, code that runs can do whatever it likes, including
        privileged operations such as issuing I/O requests and executing
        restricted instructions.
\end{itemize}

All modern hardware provides the ability for user programs to perform a 
\textbf{system call}. System calls allow the kernal to carfully expose
certain key pieces of functionality to user programs such as accessing the file
system, creating and destroying the processes, communicationg with other
processes, and allocating more memory.

\paragraph{System calls}

To execute a system call, a program must execute a special \textbf{trap}
instruction. This instruction jumps into the kernal and raises privilege level
to kernal mode; once in the kernal, the system can now perform the restricted 
operations needed and thus the required work for the calling process. When 
finished, the OS calls a special \textbf{return-from-trap} instruction, which
returns into the calling user program while simultaneously reducing the 
privilege level back to user mode.\\

When executing a trap, the caller's registers must be saved in order to 
return correctly when the OS issues the return-from-trap instruction.\\

On x86, the processor will push the program counter, flags and a few other
registers onto a per-process \textbf{kernal stack}; return-from-trap
will pop these values from the stack and resume execution.

\paragraph{How does the trap know which code to run inside the OS?}

Clearly, the calling process can't specify and address to jump to; doing so 
would allow programs to jump anywhere into the kernal which is not secure. The
kernal must control what code executes upon a trap.\\

The kernal does so by setting up a \textbf{trap table} at boot time. When the
machine boots up, it does so in kernal mode. One of the first things the OS
does is to tell the hardware what code to run when certain exceptional events
occur. The OS informs the hardware of these locations of these
\textbf{trap handlers}, usually with some kind of special instruction. Once
the hardware is informed, it remembers the location of these handlers until
the machine is rebooted, and thus hardware knows what to do when system calls
and other exceptional events take place.\\

To perform the exact system call, a \textbf{system-call number} is usually 
asigned to each system call. The user code has to place the desired system-call
number in a register or at a specific location in the stack; the OS, when
handeling the system call inside the trap handler, examines the number,
ensures it is valid, and, if it is, executes the corresponding code. This
level of indirection serves as a form of \textbf{protection}; user code cannot
specify an exact address to jump to, but rather must request a particular 
service via a number.\\

Telling the hardware where the trap tables are is a \textbf{privileged} 
operationg, otherwise you could make your own trap tables and compromise the
system.\\

How all this works:

\begin{enumerate}
    \item At boot time, the kernal initializes the trap table, and the CPU
        remembers its location for subsequent use. The kernal does so via
        a privileged instruction.
    \item The kernal sets up a few things (allocating memory, etc.) before
        using a return-from-trap instruction to start the execution of the 
        process. When the process wishes to issue a system call, it traps back
        into the OS, which handles it and once again returns control via a 
        return-from-trap to the process. The process completes its work and 
        returns from main().
\end{enumerate}

\subsubsection{Problem \#2: Switching Between Processes}

When a process is running on the CPU, this by definition means the OS is
\textit{not} running. If OS isnt running how can
it change a process? (it cant).\\

\textbf{The Crux: How to regain control of the CPU?}

How can the operating system \textbf{regain control} of the CPU so that it can
switch between processes?

\paragraph{Cooperative Approach: Wait for system calls}

In this approach, the OS regains control of the CPU by waiting for a 
system call or an illegal operation of some kind to take place, as
control is passed to the OS during these exceptions.

\paragraph{Non-Cooperative Approach: The OS takes control}

If we get stuck in an infinite loop in the cooperative approach, the only way
out is to reboot the machine.\\

\textbf{The Crux: How to gain control without cooperation} 

How can the OS gain control of the CPU even if processes are not being
cooperative? What can the OS do to ensure a rogue process does not take over
the machine?\\

\textbf{Timer interrupt:} A timer device is programmed to raise an interrupt
after a fixed delay; when the interrupt is raised, the currently running
process is halted, and a pre-configured \textbf{interrupt handler} in the OS
runs. Now the OS has regained control of the CPU, and thus can stop the 
current process and start a different one.\\

At boot time, the OS starts the interrupt timer. This is a privileged
operation.

\paragraph{Saving and Restoring Context}

The OS has regained control now. It can decide to switch or not. This decision
is made by the \textbf{scheduler}.\\

If the decision is to switch, the OS executes a \textbf{context switch}. A
context switch saves register values of the currently-executing process and
restores register values for the soon-to-be-executing process. After this, 
the OS executes a return-from-trap instruction and executes the new process.\\

By switching stacks pointers (stack pointer is a register),
the kernel enters call to the switch code in the 
context of one process (the interrupted one) amd returns in the context of
another (the soon-to-be-executing one). By switching stack is actually how the
process is switched.\\

There are two types of register saves/restores that happen here:

\begin{itemize}
    \item When the timer interrupt occurs. The \textit{user registers} of the
        running process are implicitly saved by the \textit{hardware}, using
        the kernal stack of the process.
    \item When OS performs the context switch. The \textit{kernal registers} 
        are explicitly saved by the \textit{software} (i.e. th OS), but this
        time into memory in the process structure of the process.
\end{itemize}


